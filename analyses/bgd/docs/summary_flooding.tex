% Options for packages loaded elsewhere
\PassOptionsToPackage{unicode}{hyperref}
\PassOptionsToPackage{hyphens}{url}
%
\documentclass[
]{article}
\usepackage{lmodern}
\usepackage{amsmath}
\usepackage{ifxetex,ifluatex}
\ifnum 0\ifxetex 1\fi\ifluatex 1\fi=0 % if pdftex
  \usepackage[T1]{fontenc}
  \usepackage[utf8]{inputenc}
  \usepackage{textcomp} % provide euro and other symbols
  \usepackage{amssymb}
\else % if luatex or xetex
  \usepackage{unicode-math}
  \defaultfontfeatures{Scale=MatchLowercase}
  \defaultfontfeatures[\rmfamily]{Ligatures=TeX,Scale=1}
\fi
% Use upquote if available, for straight quotes in verbatim environments
\IfFileExists{upquote.sty}{\usepackage{upquote}}{}
\IfFileExists{microtype.sty}{% use microtype if available
  \usepackage[]{microtype}
  \UseMicrotypeSet[protrusion]{basicmath} % disable protrusion for tt fonts
}{}
\makeatletter
\@ifundefined{KOMAClassName}{% if non-KOMA class
  \IfFileExists{parskip.sty}{%
    \usepackage{parskip}
  }{% else
    \setlength{\parindent}{0pt}
    \setlength{\parskip}{6pt plus 2pt minus 1pt}}
}{% if KOMA class
  \KOMAoptions{parskip=half}}
\makeatother
\usepackage{xcolor}
\IfFileExists{xurl.sty}{\usepackage{xurl}}{} % add URL line breaks if available
\IfFileExists{bookmark.sty}{\usepackage{bookmark}}{\usepackage{hyperref}}
\hypersetup{
  pdftitle={Spatio-temporal flood extent estimation in Bangladesh},
  pdfauthor={By MapAction and the Centre for Humanitarian Data},
  hidelinks,
  pdfcreator={LaTeX via pandoc}}
\urlstyle{same} % disable monospaced font for URLs
\usepackage[margin=1in]{geometry}
\usepackage{longtable,booktabs}
\usepackage{calc} % for calculating minipage widths
% Correct order of tables after \paragraph or \subparagraph
\usepackage{etoolbox}
\makeatletter
\patchcmd\longtable{\par}{\if@noskipsec\mbox{}\fi\par}{}{}
\makeatother
% Allow footnotes in longtable head/foot
\IfFileExists{footnotehyper.sty}{\usepackage{footnotehyper}}{\usepackage{footnote}}
\makesavenoteenv{longtable}
\usepackage{graphicx}
\makeatletter
\def\maxwidth{\ifdim\Gin@nat@width>\linewidth\linewidth\else\Gin@nat@width\fi}
\def\maxheight{\ifdim\Gin@nat@height>\textheight\textheight\else\Gin@nat@height\fi}
\makeatother
% Scale images if necessary, so that they will not overflow the page
% margins by default, and it is still possible to overwrite the defaults
% using explicit options in \includegraphics[width, height, ...]{}
\setkeys{Gin}{width=\maxwidth,height=\maxheight,keepaspectratio}
% Set default figure placement to htbp
\makeatletter
\def\fps@figure{htbp}
\makeatother
\setlength{\emergencystretch}{3em} % prevent overfull lines
\providecommand{\tightlist}{%
  \setlength{\itemsep}{0pt}\setlength{\parskip}{0pt}}
\setcounter{secnumdepth}{-\maxdimen} % remove section numbering
\usepackage{booktabs}
\usepackage{longtable}
\usepackage{array}
\usepackage{multirow}
\usepackage{wrapfig}
\usepackage{float}
\usepackage{colortbl}
\usepackage{pdflscape}
\usepackage{tabu}
\usepackage{threeparttable}
\usepackage{threeparttablex}
\usepackage[normalem]{ulem}
\usepackage{makecell}
\usepackage{xcolor}
\ifluatex
  \usepackage{selnolig}  % disable illegal ligatures
\fi

\title{Spatio-temporal flood extent estimation in Bangladesh}
\author{By MapAction and the Centre for Humanitarian Data}
\date{2021-03-20}

\begin{document}
\maketitle

{
\setcounter{tocdepth}{4}
\tableofcontents
}
\hypertarget{introduction}{%
\subsection{Introduction}\label{introduction}}

This report provides a summary of the results and validation of an
analysis of Sentinel-1 satellite imagery to estimate flood extent over
time in Bangladesh from June - August 2020. This analysis is focused on
flooding within five selected districts: Bogra, Gaibandha, Jamalpur,
Kurigram and Sirajganj.

\hypertarget{processing-sentinel-1-data-to-derive-flood-extents}{%
\subsection{Processing Sentinel-1 data to derive flood
extents}\label{processing-sentinel-1-data-to-derive-flood-extents}}

In collaboration with colleagues from MapAction we took freely available
European Space Agency's
\href{https://sentinel.esa.int/web/sentinel/user-guides/sentinel-1-sar}{Sentinel-1
Synthetic Aperture Radar (SAR) imagery}, which provides images with 10m
resolution, to estimate the flood extent over time. Sentinel-1 SAR data
has been frequently applied to map flooding in past literature, even
specifically in Bangladesh. In addition to being freely available, SAR
imagery is particularly useful for flood mapping as it can be captured
even in the presence of cloud cover, unlike imagery from optical sensors
such as Landsat and MODIS. This is particularly relevant in areas such
as Bangladesh which have significant cloud cover during flooding
seasons. Water bodies can be identified from SAR imagery due to their
dark appearance.

The methodology used is adapted from the
\href{http://www.un-spider.org/advisory-support/recommended-practices/recommended-practice-google-earth-engine-flood-mapping/in-detail}{UN-SPIDER
Knowledge portal} and applies a change detection and thresholding (CDAT)
approach to identify flooded areas. A CDAT methodology for identifying
flooded areas from Sentinel-1 data has been successfully applied in
contexts such as Bangladesh, Namibia and the UK. We performed the
analysis in Google Earth Engine, which provides easy access to
Sentinel-1 data and allows for fast, cloud-based data processing. The
image processing methodology described below is largely summarized from
the UN-SPIDER guidance.

The estimates of flood extent were then aggregated to a given admin unit
(mauzas, in this case) by calculating the total flooded fraction within
each unit for each point in time. Note that the area of permanent water
bodies (as identified by the JRC Global Surface Water dataset) was
removed from the area of each admin unit. The flooded fraction values
thus represent the fraction of flooded area that is not normally covered
by water.

More details on satellite imagery processing

\hypertarget{imagery-filtering-and-preprocessing}{%
\paragraph{Imagery filtering and
preprocessing}\label{imagery-filtering-and-preprocessing}}

Available Sentinel-1 imagery for the time period of interest is filtered
according to the instrument mode, polarization, and pass direction. This
filtering is necessary to ensure that mosaicked images share the same
characteristics. Table 1 below briefly outlines each of these
parameters. The selected imagery has already undergone preprocessing
steps to convert pixel values to their backscatter coefficient. These
steps are detailed in
\href{https://developers.google.com/earth-engine/guides/sentinel1}{this
page} and include thermal noise removal, radiometric calibration, and
terrain correction. In addition, this methodology applies a smoothing
filter to the imagery to reduce the speckle effect of radar imagery.

\begin{longtable}[]{@{}lll@{}}
\toprule
\begin{minipage}[b]{(\columnwidth - 2\tabcolsep) * \real{0.05}}\raggedright
Filtering parameter\strut
\end{minipage} &
\begin{minipage}[b]{(\columnwidth - 2\tabcolsep) * \real{0.35}}\raggedright
Possible values\strut
\end{minipage} &
\begin{minipage}[b]{(\columnwidth - 2\tabcolsep) * \real{0.60}}\raggedright
Description\strut
\end{minipage}\tabularnewline
\midrule
\endhead
\begin{minipage}[t]{(\columnwidth - 2\tabcolsep) * \real{0.05}}\raggedright
Instrument mode\strut
\end{minipage} &
\begin{minipage}[t]{(\columnwidth - 2\tabcolsep) * \real{0.35}}\raggedright
IW, EW, SM\strut
\end{minipage} &
\begin{minipage}[t]{(\columnwidth - 2\tabcolsep) * \real{0.60}}\raggedright
\href{https://sentinel.esa.int/web/sentinel/user-guides/sentinel-1-sar/acquisition-modes}{Notes
on acquisition modes}. We selected data from the `IW' acquisition mode,
as it is cited as the main acquisition mode for monitoring land
changes.\strut
\end{minipage}\tabularnewline
\begin{minipage}[t]{(\columnwidth - 2\tabcolsep) * \real{0.05}}\raggedright
Polarization\strut
\end{minipage} &
\begin{minipage}[t]{(\columnwidth - 2\tabcolsep) * \real{0.35}}\raggedright
HH+HV, VV+VH, VV, HH\strut
\end{minipage} &
\begin{minipage}[t]{(\columnwidth - 2\tabcolsep) * \real{0.60}}\raggedright
Horizontal and vertical. We selected the `VH' mode as it is indicated by
the UN-SPIDER guidance as the most optimal for flood mapping. However
sources have also noted that the `VV' polarization may produce more
accurate results in some instances\strut
\end{minipage}\tabularnewline
\begin{minipage}[t]{(\columnwidth - 2\tabcolsep) * \real{0.05}}\raggedright
Pass direction\strut
\end{minipage} &
\begin{minipage}[t]{(\columnwidth - 2\tabcolsep) * \real{0.35}}\raggedright
Ascending or descending\strut
\end{minipage} &
\begin{minipage}[t]{(\columnwidth - 2\tabcolsep) * \real{0.60}}\raggedright
Direction of orbit. We performed the analysis using data from both pass
directions (only comparing images from the same direction).\strut
\end{minipage}\tabularnewline
\begin{minipage}[t]{(\columnwidth - 2\tabcolsep) * \real{0.05}}\raggedright
Relative orbit\strut
\end{minipage} &
\begin{minipage}[t]{(\columnwidth - 2\tabcolsep) * \real{0.35}}\raggedright
This value is dependent on the location of the satellite orbit for the
area of interest. Values can be derived
\href{https://gis.stackexchange.com/questions/237116/sentinel-1-relative-orbit\#:~:text=You\%20actually\%20can\%20find\%20the,\%2D\%2073\%2C\%20175\%20\%2B\%201}{according
to the imagery filename}.\strut
\end{minipage} &
\begin{minipage}[t]{(\columnwidth - 2\tabcolsep) * \real{0.60}}\raggedright
Filtering by relative orbit ensures that one is comparing images with
the same viewing geometry.\strut
\end{minipage}\tabularnewline
\bottomrule
\end{longtable}

\hypertarget{change-detection-and-thresholding-cdat-to-identify-flooding}{%
\paragraph{Change detection and thresholding (CDAT) to identify
flooding}\label{change-detection-and-thresholding-cdat-to-identify-flooding}}

This methodology identifies flood extent by comparing between
before-flooding and after-flooding imagery mosaics for the area of
interest. In this case, we took the average of all images from December
2019 to January 2020 from the area of interest to generate the baseline
before-flooding mosaic. We also checked the EM-DAT database to ensure
that there was not any recorded flooding during this period.

The after-flood mosaic is divided by the before-flooding mosaic, with
pixel intensity in the resulting image indicating the degree of change
between the two images. A threshold of 1.25 is applied to generate a
binary layer indicating the full estimated extent of flooding. This
threshold level is taken directly from the UN-SPIDER guidance, where it
was selected `through trial and error'. The appropriateness of this
threshold level was also manually checked by comparing the derived flood
extents with the after-flooding satellite imagery for selected dates.
These parameters influencing this component of the analysis are
summarized in the table below.

\begin{longtable}[]{@{}lll@{}}
\toprule
\begin{minipage}[b]{(\columnwidth - 2\tabcolsep) * \real{0.09}}\raggedright
Parameter\strut
\end{minipage} &
\begin{minipage}[b]{(\columnwidth - 2\tabcolsep) * \real{0.07}}\raggedright
Value\strut
\end{minipage} &
\begin{minipage}[b]{(\columnwidth - 2\tabcolsep) * \real{0.84}}\raggedright
Description\strut
\end{minipage}\tabularnewline
\midrule
\endhead
\begin{minipage}[t]{(\columnwidth - 2\tabcolsep) * \real{0.09}}\raggedright
Non-flooded reference period\strut
\end{minipage} &
\begin{minipage}[t]{(\columnwidth - 2\tabcolsep) * \real{0.07}}\raggedright
01-12-2019 to 31-01-2020\strut
\end{minipage} &
\begin{minipage}[t]{(\columnwidth - 2\tabcolsep) * \real{0.84}}\raggedright
Following
\href{https://www.sciencedirect.com/science/article/abs/pii/S0924271620301702?}{past
work in Bangladesh}, we took the median values of imagery from December
2019 to January 2020 to generate two non-flooded reference image mosaics
(one each for ascending and descending pass directions). Each reference
mosaic was generated using 10 images for the area of interest.\strut
\end{minipage}\tabularnewline
\begin{minipage}[t]{(\columnwidth - 2\tabcolsep) * \real{0.09}}\raggedright
Flood extent threshold\strut
\end{minipage} &
\begin{minipage}[t]{(\columnwidth - 2\tabcolsep) * \real{0.07}}\raggedright
1.25\strut
\end{minipage} &
\begin{minipage}[t]{(\columnwidth - 2\tabcolsep) * \real{0.84}}\raggedright
From UN-SPIDER guidance, where it was selected through trial and
error.\strut
\end{minipage}\tabularnewline
\bottomrule
\end{longtable}

The flood extent output is further refined to mask the main water bodies
and also remove regions where the average slope is greater than 10\%.
Main water bodies are identified using the
\href{https://global-surface-water.appspot.com/}{JRC Global Surface
Water dataset}, using a threshold of areas covered by water for at least
10 months in a year. Slope is calculated from the
\href{https://developers.google.com/earth-engine/datasets/catalog/WWF_HydroSHEDS_03VFDEM}{WWF
HydroSHEDS DEM}, based on SRTM data.

To understand the evolution of flooding over time, we repeated this
change detection process separately on all available Sentinel-1 data for
the area of interest between June - August 2020. In this case, 17
mosaicked images were available throughout this time period to cover our
area of interest, generating a total of 17 output Shapefiles that
delineate flood extent for dates between June - August 2020.

\hypertarget{modelling-flooding-over-time-with-parametric-curve-fitting}{%
\subsection{Modelling flooding over time with parametric curve
fitting}\label{modelling-flooding-over-time-with-parametric-curve-fitting}}

some test

\end{document}
